%%%%%%%%%%%%%%%%%%%%%%%%
%
% $Author: Raunak Shahare $
% $Datum: 2023-06-29  $
% $Pfad: BA23-02-Sales-Predictor/report/Contents/en/ToDo.tex $
% $Version: 1.0 $
% $Reviewed by: Deepti and Sadegh  $
% $Review Date: 2023-06-30 $
%
%%%%%%%%%%%%%%%%%%%%%%%%




\chapter{To Do}


	\section{To-Do List for Sales Prediction Project using Machine Learning}
	
	\begin{enumerate}
		\item Define the project scope and objectives: Clearly outline the goals and objectives of the sales prediction project. Determine the specific aspects of sales you want to predict, such as total sales volume, revenue, or product demand.
		
		\item Gather and clean the data: Identify the relevant data sources for your sales prediction project. This may include historical sales data, customer data, marketing data, economic indicators, and any other relevant information. Clean and preprocess the data to ensure it is suitable for analysis.
		
		\item Explore and visualize the data: Conduct exploratory data analysis to gain insights into the data. Use statistical techniques and visualizations to understand patterns, trends, and relationships within the data. Identify any missing data or outliers that need to be addressed.
		
		\item Split the data into training and testing sets: Divide the dataset into two parts: a training set and a testing set. The training set will be used to train the machine learning model, while the testing set will be used to evaluate its performance.
		
		\item Select and train a machine learning model: Choose an appropriate machine learning algorithm for sales prediction. This could include regression models (such as linear regression, decision trees, random forests, or gradient boosting), time series forecasting models (such as ARIMA or LSTM), or other models suitable for sales prediction. Train the selected model using the training dataset.
		
		\item Evaluate the model's performance: Use the testing dataset to assess the performance of the trained model. Calculate relevant evaluation metrics such as mean squared error (MSE), mean absolute error (MAE), or R-squared to measure the accuracy of the predictions. Make adjustments or consider trying alternative models if the performance is not satisfactory.
		
		\item Fine-tune the model: Optimize the model's hyperparameters to improve its performance. Use techniques such as cross-validation, grid search, or random search to find the optimal combination of hyperparameters for your chosen model.
		
		\item Implement the model in a production environment: Once you are satisfied with the model's performance, deploy it in a production environment. This may involve integrating the model into an existing software system or creating a user interface to make predictions accessible to stakeholders.
		
		\item Monitor and update the model: Continuously monitor the performance of the deployed model and collect new data as it becomes available. Periodically retrain the model using updated data to ensure it remains accurate and up to date.
		
		\item Document the process and findings: Maintain detailed documentation throughout the project, including data preprocessing steps, model selection, training, and evaluation. Document the key findings, insights, and recommendations from the sales prediction project.
		
		\item Communicate results and insights: Present the results and insights derived from the sales prediction model to stakeholders, such as sales managers or executives. Clearly explain the limitations and assumptions of the model and provide recommendations based on the predictions generated.
		
		\item Iterate and improve: Use the feedback from stakeholders and users to improve the model further. Incorporate any new data sources or features that may enhance the accuracy and usefulness of the sales predictions.
	\end{enumerate}
	



%%%%%%%%%%%%%%%%%%%%%%%%
%
% $Author: Sadegh Naderi $
% $Datum: 2023-05-02  $
% $Pfad: BA23-02-Sales-Predictor/report/Contents/en/DomainKnowledge.tex $
% $Version: 1.0 $
% $Reviewed by: Deepti, Sadegh and Raunak $
% $Review Date: 2023-05-05 $
%
%%%%%%%%%%%%%%%%%%%%%%%%

\chapter{Domain Knowledge}

\section{Introduction}

Sales forecasting is an essential tool that helps businesses predict the future demand for their products and services. This information can be used to make strategic decisions, such as setting sales targets, determining pricing strategies, and managing inventory levels. In this report, we will provide an overview of the key concepts, methods, and tools used in sales forecasting.

\section{Demand Forecasting}

Demand forecasting is the process of estimating the future demand for a company's products or services. This can be done using quantitative methods, such as time-series analysis and regression analysis, or qualitative methods, such as market research and expert opinions. The accuracy of demand forecasting depends on various factors, such as the quality and availability of data, the complexity of the market, and the accuracy of the forecasting models used.

\section{Sales Forecasting Methods}

There are several methods used in sales forecasting, including the following:
	\begin{enumerate}
		\item Historical Sales Data Analysis: This method involves analyzing past sales data to identify patterns and trends that can be used to forecast future sales. This method is useful when historical data is available, and the market conditions are stable.
		\item Market Research: This method involves conducting surveys, focus groups, and other research methods to gather information about customer preferences and market trends. This method is useful when there is limited historical data available, or when the market conditions are rapidly changing.
		\item Expert Opinion: This method involves seeking input from industry experts and sales professionals to gain insight into market trends and customer behavior. This method is useful when there is limited historical data available, or when the market conditions are rapidly changing.
		\item Regression Analysis: This method involves analyzing the relationship between sales and various factors, such as price, promotion, and distribution. This method is useful when there is a significant amount of historical data available, and when the market conditions are stable.
	\end{enumerate}

\section{Sales Forecasting Tools}

There are several tools used in sales forecasting, including the following:
	\begin{enumerate}
		\item Excel Spreadsheets: Excel spreadsheets are commonly used for simple sales forecasting tasks, such as calculating the average monthly sales or identifying seasonal trends.
		\item Statistical Software: Statistical software, such as SPSS and SAS, are used for more complex sales forecasting tasks, such as time-series analysis and regression analysis.
		\item Forecasting Software: Forecasting software, such as Forecast Pro and Oracle Crystal Ball, are specialized software tools designed for sales forecasting tasks. These tools provide advanced features, such as automated model selection, forecasting accuracy measures, and what-if scenario analysis.
	\end{enumerate}

\section{Seasonality and Oil Prices in Ecuador}

Seasonality and oil prices are two significant factors that can impact sales forecasting in Ecuador. Ecuador's economy is heavily dependent on the agriculture sector, and seasonal variations in weather patterns can significantly impact sales. For instance, the country's flower industry, which is one of the largest in the world, experiences high demand during Valentine's Day and Mother's Day. Similarly, the shrimp industry experiences high demand during Christmas and New Year. As a result, businesses in these industries need to adjust their sales forecasting models to account for seasonal variations in demand.

Ecuador is a significant oil-producing country, and changes in oil prices can have a significant impact on the country's economy and sales forecasting. When oil prices are high, the government has more revenue to invest in infrastructure and other projects, which can stimulate economic growth and increase consumer spending. On the other hand, when oil prices are low, the government may need to cut back on spending, leading to a slowdown in economic growth and decreased consumer spending. Businesses in Ecuador that are heavily dependent on oil revenues, such as the transportation industry, need to factor in the impact of oil prices on sales forecasting models.

Businesses in Ecuador need to consider the impact of seasonality and oil prices on sales forecasting. By using appropriate forecasting methods and tools, businesses can adjust their strategies and make informed decisions to optimize their operations and increase their competitiveness in the market.
